\section{Beschreibung der Arbeit / Disposition}
\hspace{-0.9cm} % um die Tabelle linksbündig zu haben (ging mit flushleft nicht)
\begin{tabular}{p{4,5cm} p{10cm}} 
Name des Schreibenden: & Till J. ERNST \\ 
Name des Referenten: & Gregor WALLER \\
Vertiefungsrichtung: & Klinische Psychologie \\
& \\
Titel der Arbeit (prov.): & Beeinflusst das Medienverhalten der Eltern das subjektive Wohlbefinden? Der Einfluss des Bindungsstils und Stress auf das Medienverhalten der Eltern im Beisein ihrer Kinder und dessen Auswirkung auf das Subjektive Wohlbefinden. \\
\end{tabular} \\
% Fragestellung und Annahmen
\subsection{Fragestellung und Annahmen}
\subsubsection{Fragestellung}
Welchen Einfluss hat der Bindungsstil und das aktuelle Stressempfinden der Eltern auf das im Beisein der Kinder praktizierte Medienverhalten und hat dieses Verhalten Auswirkungen auf das elterliche subjektive Wohlbefinden?
\subsubsection{Annahmen}
Der Bindungsstil und das aktuelle Stressempfinden der Eltern wirkt sich auf deren Medienverhalten aus. Dieses Verhalten wiederum hat Auswirkungen auf das subjektives Wohlbefinden der Eltern.

Der Bindungsstil hat einen Einfluss auf das Medienverhalten \cite{Lin2015, Monacis2017}. Weitere Studien konnten aufzeigen, dass zum Beispiel Internetsucht mit einem unsicherer \cite{Lin2011, Severino2013}, mit einem ängstlich vermeidenden \cite{Shin2011} und mit einem unsicher distanzierten, sowie unsicher ambivalenten Bindungsstil \cite{Odaci2014} einhergeht. Ein unsicherer Bindungsstil und eine problematische Internetnutzung wiederum beeinflusst das subjektive Wohlbefinden \cite{Odaci2014}.	 
Dadurch wird angenommen, dass Eltern mit einem unsicheren Bindungsstil mehr Zeit im Beisein ihrer Kinder mit Medien verbringen und diese in einem ungesunden (pathologischen) Ausmass nutzen, wodurch sich dieser Umgang negativ auf das subjektive Wohlbefinden niederschlägt.  

Die Annahme, dass sich Stress auf das Medienverhalten auswirkt, wird von der Studie um \citeA{Mark2014} erhärtet. Daraus geht hervor, dass sich Stress auf die Nutzungsdauer vor dem Computer auswirkt und Stress in direktem Zusammenhang mit Multitasking steht. Berufsgruppen, die besonders häufig von Multitasking-Anforderungen betroffen sind, befinden sich in den Bereichen Erziehung- und Unterricht und im Gesundheits- und Sozialwesen \cite{Lohmann2012}. Dies lässt darauf hindeuten, dass Eltern, die sich mit der Erziehung der Kinder beschäftigen, besonders vom Multitasking betroffen sind und dementsprechend anfällig für Stress sind. Multitasking eignet sich auch für bestimmte Aufgaben besser als für andere. Je weniger Ressourcen die Parallelaufgabe benötigt und je ähnlicher und automatisierbarer sie sind, desto eher wird Multitasking angewendet \cite[S.~10]{Zimber2016}. Es wird angenommen, dass während der Betreuung der Kinder viele solcher Aufgaben anfallen, bei denen die Versuchung Multitasking angewendet werde soll gegeben ist. Somit könnte eine erhöhte Stressbelastung, in einer erhöhten Ausübung von Multitasking und somit zu einer verstärkten Mediennutzung führen.


TBD: Medienverhalten und die Auswirkung auf das SWB -> mehr Medienkonsum = schlechteres SWB


\subsection{Hypothesen (Statistische Verfahren)}
\subsubsection{Haupthypothesen}
\subsubsection{Arbeithypothesen}
TBD
\subsection{Art der Arbeit}
Empirische Querschnitts-Studie mit quantitativem Charakter.
% Theoretischer Hintergrund
% -------------------------
\subsection{Theoretischer Hintergrund / Stand der Forschung}
\subsubsection{Einführung und allgemeiner Medienumgang Eltern / Kind}
Seit der Erscheinung des Internets und der Digitalisierung taucht die Frage auf, was für Auswirkungen diese neuen Technologien auf die Benutzer haben. Gemäss heutiger Forschung kann aufgezeigt werden, dass immer mehr jüngere Kinder sich mit dem Internet und den neuen Medien beschäftigen \cite{Rideout2013, Chaudron2015}, obwohl es ihnen gemäss \citeA{Lobe2011} an technischen, kritischen und sozialen Fähigkeiten mangelt. Dabei stellt sich die Frage, was für Auswirkungen neue Medien auf die Kinder haben \cite{Tomopoulos2010, Pempek2014, Livingstone2015, Masur2015, Troseth2016}. 

Die meisten Umfragen zum Mediennutzungsverhalten von Kindern wurden im Alter zwischen 9 und 16 Jahren durchgeführt \cite{Chaudron2015}. Aus dem amerikanischen Report \citeA{Rideout2013} geht hervor, dass der Zugang zu mobilen Geräten bei Kindern von 8 Jahren und jünger in Amerika gegenüber 2011 von 8\% beim iPad auf 40\% im Jahre 2013 angestiegen ist. Der Zugang zu einem Smart-Device von 52\% auf 75\% gestiegen ist. Gemäss diesem Report hatten 38\% aller Kinder unter 2 Jahren bereits ein Mobilgerät für die Nutzung von Medien benutzt (gegenüber 10\% in 2011). Studien in der EU kommen in auf ähnliche Ergebnisse \cite{Holloway2013}: Zunahmen von Internetkonsum bei Kindern unter 9 Jahren; Kinder unter 9 Jahren erfreuen sich bei diversen online Aktivitäten wie Video schauen, Gamen, Informationssuche, Aufgaben erledigen und mit anderen Kindern sozialisieren; eine Zunahme bei der Verwendung von Geräten mit Touchscreen bei Kindern im Vorschulalter und Kleinkindern ist zu beobachten. Zudem erwähnen die Autoren den digitalen Footprint, der bereits bei sehr kleinen Kindern vorhanden ist. Dieser wird in den meisten Fällen von den eigenen Eltern erzeugt (Fotos teilen, Blogs schreiben, Videos teilen, etx.).
Der Medienkonsum von Kindern und Jugendlichen in der Schweiz wurde durch die ZHAW in der MIKE- und der JAMES-Studie \cite{Suter2015, Waller2016} erhoben. In der MIKE-Studie wurden 2014/2015 Kinder im Primarschulalter (zwischen 6 und 13 Jahren) zu ihrem Medinenutzungsverhalten befragt. Die JAMES-Studie, welche seit 2010 im Zweijahresrythmus durchgeführt wird, erfasste 2016 repräsentative Zahlen zur Mediennutzung von Jugendlichen zwischen 12 und 19 Jahren. In der Schweiz fehlen Studien im Umgang mit Medien im Bereich Eltern und deren Kleinkinder. Die Studie von \citeA{Livingstone2015a} untersucht den Umgang der Eltern mit digitalen Medien und wie sie diese den Kindern vermitteln wobei der soziökonomische Status, wie Einkommen und Bildung, die digitale Mediennutzung im Umgang mit ihren Kindern beeinflusst. Länderübergreifende Studien konnten Unterschiede im Verhalten der Eltern im Umgang mit digitalen Medien feststellen \cite{Helsper2013}. Die gemeinsame Nutzung von Medien zwischen Eltern und Kindern wurde unter anderem von \citeA{Livingstone2008, Nikken2014, Plowman2014, Connell2015, Vaala2015, Harrison2015} untersucht. 

Die Auswirkungen von Medienkonsum kann nicht abschliessend beantwortet werden. Es scheint, als ob zum Beispiel die Zeit, die Kinder vor einem Bildschirm sind, abhängig von der Interaktionsfaktoren zwischen Eltern und Kindern ist, zusätzlich könnte dies in hohem Mass von der Einstellungen der Eltern abhängen \cite{Lauricella2015}. Der direkte Vergleich von einem digitalen Medium (TV) und einem analogen (Buch) zeigte, dass sich die Kommunikation zwischen der Mutter und ihrem Kind beim Lesen lernen währendem ein TV im Hintergrund läuft verschlechterte \cite{Nathanson2011}.

Es benötigt weitere Studien, die sich diesem Thema annehmen \cite{Wartella2016}. Die Frage, wie Eltern ihr Kinder bezüglich Kreativität, Lernen und Entwicklung im Bezug zum Medienkonsum ist unzureichend beantwortet und benötigt weitere Forschung \cite{AmericanAcademyofPediatrics2011,Troseth2016}. 
% Bindungsmuster
\subsubsection{Einfluss der Bindungsmuster auf das Medienverhalten} 
Die Familie ist das erste Umfeld, in dem sich ein Kind befindet  und das als Prototyp für zukünftige Beziehungen und Interaktionen fungiert \cite{Floros2013}. Die Familie ist in erster Linie für das Stillen der kindlichen Grundbedürfnisse verantwortlich (nicht nur materiell sondern auch psychologisch). Die psychologischen Bedürfnisse werden durch soziale Bedürfnisse befriedigt \cite{Hazan1994}. Die elterliche Responsiveness dient beim Kind dazu, seine internen Arbeitsmodelle und seinen Bindungsstil gemäss John Bowlby und Mary Ainsworth zu entwickeln \cite{Bretherton1999}. Ein Kind kann einen sicheren Bindungsstil entwickeln, wenn die Hauptbezugsperson die Signale des Kindes ohne Verzögerung wahrnimmt, sie richtig interpretiert, sie angemessen befriedigt und sie prompt befriedigt \cite{Bell1972}. Studien konnten zeigen, dass sich eine problematische Mutter-Kind-Beziehung auf die spätere Kommunikations- und Beziehungspräferenz niederschlägt \cite{Szwedo2011}. Die Studie von \citeA{Lin2015} konnte aufzeigen, dass der Bindungsstil bei der Nutzung von sozialen Netzwerken einen signifikanten Faktor bezüglich der sozialen Beziehungsorientierung bei Facebook und einen Einfluss auf das soziale Kapital ausmacht. \citeA{Monacis2017} konnte einen negativen Zusammenhang mit dem sicheren Bindungsstil und Onlinesucht herstellen.  

Gemäss \citeA{Fraley2000} werden im englisch sprechenden Raum vier verbreitete Selbsterfassungsfragebögen für die Erfassung des Bindungsstils (engl. Attachment) eingesetzt: Der \textit{
Experience in Close Relationship scales (ECR)} \cite{Brennan1998}, der \textit{Adult Attachment Scales (AAS)} \cite{Collins1990}, der \textit{Relationship Styles Questionnaire (RSQ)} \cite{Griffin1994} und der von \citeA{Simpson1996} entwickelte \textit{Adult Attachment Questionnaire (AAQ)}. Im deutschsprachigen Raum wird unter anderem die von \citeA{Grau1999} entwickelte \textit{Skala zur Erfassung von Bindungsrepräsentationen} in Paarbeziehungen, die \textit{Beziehungsspezifischen Bindungsskalen für Erwachsene (BBE)} zur Erfassung des partnerschaftlichen Bindungsstils von \citeA{Asendorpf1997} und der ins deutsche übersetzte \textit{Revised Adult Attachment Scale} von \citeA{Collins1990} eingesetzt. 

% Stress
\subsubsection{Einfluss von Stress auf das Medienverhalten}
Gemäss \citeA{Mark2014} steht Stress in direktem Zusammenhang mit Multitasking. Der Begriff Multitasking ist wissenschaftlich noch nicht einheitlich definiert worden. Multitasking steht in der Informatik für die Fähigkeit eines Betriebssystems, verschiedene Aufgaben parallel ausführen zu können  und ob der Mensch mehrere Aufgaben simultan erledigen kann, wird von Neurowissenschaftlern angezweifelt \cite{Zimber2016}. 

Im Rahmen der Risikofaktoren von Problemverhalten bei Kindern wird häufig vom Faktor Stress bei den Eltern gesprochen, der eng mit dem Verhalten in Erziehungssituationen gezeigt wird und über längere Zeit ungünstige Folgen für das Individuum sowohl dessen Umfeld aufweist \cite{Cina2009}. Psychische, physische sowohl soziale Störungen stehen im Zusammenhang mit Stress \cite{Elfering2002, Burisch1994}. Insbesondere die engen Familienmitglieder sind oft direkt oder Indirekt von den Auswirkungen des Stresses betroffen. So zeigen Studien einen Zusammenhang zwischen Stress und schlechtem psychischen Befinden \cite{Burisch1994, Krohne1997}, einer negativen Partneschaftswualität \cite{Bodenmann2000, Bodenmann1999, Bodenmann2000a} und ungünstigem Erziehungsverhalten \cite{Abidin1992, Belsky1984, WebsterStratton2000}. Tägliche Widrigkeiten scheinen Auswirkungen auf das Erziehungsverhalten der Eltern in Form eines negativen und aversiven Bindunsstils \cite{Dumas1989, Webster-Stratton1988} und einer geringen emotionalen Verfügbarkeit für die Kinder \cite{Campbell1991} zu haben. Es wird angenommen, dass durch Stress ein erhöhter Medienkonsum stattfinden wird und dieser weite Bereiche des Familienlebens beeinflusst.

Für die Messung des Stressniveaus werden im englisch sprachigen Raum drei verbreitete Instrumente eingesetzt: Der \textit{Stress Appraisal Measure (SAM)} der \textit{Impact of Event Scale (IES)} und der \textit{Perceived Stress Scale (PSS)}. Der PSS ist für die Messung von Stress weit verbreitet \cite{Andreou2011}. Ursprünglich wurde der PSS von \citeA{Cohen1983} aus 14 Items zum wahrgenommen Stress entwickelt. Die Items nehmen Bezug auf die Häufigkeit der Gefühlen und Gedanken des vergangenen Monats des Probanden. Zudem existieren zwei Kurzversionen dieses Tests, der PSS-4 und der PSS-10 mit jeweils vier, resp. 10 Items, die aus den ursprünglich 19 Items ausgewählt wurden. Hohe PSS Werte gehen mit höheren Cortisolwerten einher, welcher als Biomarker für Stress gilt \cite{Malarkey1995, VanEck2005}. Der Test wurde in diverse Sprachen übersetzt und wurde auf dessen Konstruktvalidität geprüft \cite{Cohen1988, Byrne2005}.

und der \textit{Revised Perceived Stress Questionnaire (PSQ-R)} \cite{Levenstein1993, Fliege2005, Cohen1983}; \textit{Depressions-Angst-Stress-Skalen (DASS)} von \cite{Nilges2015}; \textit{Allgemeines Stressniveau (ASN)} von \citeA{Bodenmann2000, Gabriel2006, Cina2009}.
 
% SWB
\subsubsection{Auswirkung Medienverhalten auf das SWB}
TBD

% Methode
% -------
\subsection{Methode}
Bei dieser Masterarbeit handelt es sich um eine Querschnitts-Studie, in der mittels Fragebogen die Mediennutzung der Eltern im Beisein der Kinder, der Bindungsstil der Eltern, der aktuellen Stresslevel des betreuenden Elternteils und das subjektive Wohlbefinden des Elternteils empirisch erfasst werden sollen. 
Es werden deskriptive statistische Verfahren zur Aufdeckung von Datenstrukturen und Abhängigkeitsstrukturen von einer zu rekrutierenden Stichprobe eingesetzt.
% Untersuchungsplan
% -----------------
\subsubsection{Untersuchungsplan / Vorgehen}
TBD: Grafische Übersicht - Was wird wann gemacht.
\begin{enumerate}
    \item Disposition
    \item Literaturrecherche
    \item Zusammenstellen der Umfrage und Test
    \item Wahl des Umfragetools (Smartphone, Online-Survey)
    \item Rekrutierung der Probanden über einschlägige Organisationen (z.B. Hebammenvereinigung Schweiz, Geburtshäuser, etc.)
    \item Je nach Erhebungsinstrument Einführung der Probanden (z.B. bei Smartphone-Umfrage Konfiguration Startzeitpunkt und Funktions-Kontrolle).
    \item Daten Generierung
    \item Abschluss Umfrage, Bereinigung und Analyse der Rohwerte
    \item Auswertung der Befunde
\end{enumerate}
\subsubsection{Datenerhebung}
Für die Beantwortung der Fragestellung und die Überprüfung der Hypothesen sollen folgende Variablen erhoben werden: 
\begin{seriate}
    \item die Mediennutzung der Eltern im Beisein ihrer 0 bis zu einem Jahr alten Kinder
    \item das aktuelle subjektive Wohlbefinden der Eltern
    \item der Bindungsstil der Eltern und
    \item der aktuelle Stresslevel der Eltern.
\end{seriate}

Zudem sollen Kovariate wie 
\begin{seriate}
    \item Geschlecht
    \item Alter
    \item Nationalität
    \item Ausbildung
    \item Beruf und Anstellungsgrad
\end{seriate}
erhoben werden.

Die Erfassung dieser Daten erfolgt mittels Fragebogen, der aus einzelnen bestehenden Fragebögen, oder Teilen davon, zusammengesetzt ist. Folgende Fragebögen werden für diese Untersuchung verwendet:
Hier sollen die einzelnen Fragebögen aufgelistet werden. Wie z.B.:
\begin{enumerate}
    \item Media Use
    \item Fragebogen zur Erfassung des persönlichen Bindungsstils der Eltern: 
In dieser Arbeit wird die Skala zur Erfassung von Bindungsrepräsentationen nach \citeA{Grau1999} verwendet. Diese dient zur Klassifizierung der vier Bindungsstile nach \citeA{Bartholomew1990} anhand der beiden Dimensionen Angst und Vermeidung. Der Fragebogen besteht aus zwei intern konsistenten, voneinander unabhängigen Skalen, denen jeweils zehn Items der beiden Dimensionen Angst bzw. Vermeidung zugeordnet sind \cite{Baadte2006}. Zum Antworten dient eine sieben-stufige Skala mit den beiden Polen ‚stimmt nicht‘ (1) bzw. ‚stimmt‘ (7). Alternativ könnte der ins Deutsche übersetzte Adult Attachment Scale (AAS) von \citeA{Collins1990} verwendet werden. Dieser, in der revidierten Fassung (AAS-R),  ist über die Testzentrale des Verlags Hogrefe zu beziehen und verfügt  über 18 Items zur Erfassung grundlegender Dimensionen von Bindung.
    \item Fragebogen zur Erfassung von Stress der Bezugsperson:
    
    \item Subjektives Wohlbefinden 
\end{enumerate}

TBD: siehe \cite{AmericanAcademyofPediatrics2013} und \cite{Plowman2014}, \cite{Nikken2014}

%Stichprobe und Rekrutierung
\subsubsection{Stichprobe und Rekrutierung}
\subsubsection{Statistische Verfahren}
\subsubsection{Weitere Bedingungen}
\subsection{Abgrenzung}
In dieser Arbeit kann die Auswirkung des Medienverhaltens der Eltern auf die Kinder nicht untersucht werden. Durch die begrenzte Mitteilungsmöglichkeiten der Kinder kann die direkte Auswirkung der Medienverhaltens in diesem Rahmen nicht ermittelt werden. Mögliche längerfristige Auswirkungen auf den Bindungsstil der Kinder müsste mittels Langzeitstudie erhoben werden. 
