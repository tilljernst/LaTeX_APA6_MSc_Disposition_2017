\section{Beschreibung der Arbeit / Disposition}
% Fragestellung und Annahmen
\subsection{Fragestellung und Annahmen}
\subsubsection{Fragestellung}
Welchen Effekt hat der Bindungsstil und das aktuelle Stressempfinden der Eltern auf das im Beisein der Kinder praktizierte Medienverhalten? Kann zwischen diesem elterlichen Verhalten und deren subjektiven Wohlbefinden ein Zusammenhang gefunden werden?
\subsubsection{Annahmen}
Der Bindungsstil und das Ausmass an Stress der Eltern könnten einen Effekt auf deren Medienverhalten haben. Dieses Verhalten wiederum könnte mit dem subjektiven Wohlbefinden der Eltern zusammenhängen.

Gemäss Studien haben die Bindungsstile nach \citeA{Bowlby1969} (sicher, unsicher-vermeidend und unsicher-ambivalent) einen Effekt auf das Medienverhalten \cite{Lin2015, Monacis2017}. Beispielsweise konnte nachgewiesen werden, dass ein sicherer Bindungsstil im Umgang mit sozialen Netzwerken mit einem höheren verbindenden Onlinekapital (\textit{engl. }online bonding capital) und einer höheren Beziehungsfähigkeit offline  einhergeht \cite{Lin2015}. Weiter wirkt ein sicherer Bindungsstil protektiv gegenüber einer Abhängigkeit gegenüber online Spielen, Internet und sozialen Medien aus \cite{Monacis2017}.  
Weitere Studien konnten aufzeigen, dass zum Beispiel Internetsucht mit einem unsicheren \cite{Lin2011, Severino2013}, mit einem ängstlich vermeidenden \cite{Shin2011} und mit einem unsicher distanzierten sowie unsicher ambivalenten Bindungsstil \cite{Odaci2014} einhergeht. Ein unsicherer Bindungsstil und eine problematische Internetnutzung wiederum haben einen Effekt auf das subjektive Wohlbefinden \cite{Odaci2014}.	 
Dadurch wird angenommen, dass Eltern mit einem unsicheren Bindungsstil mehr Zeit im Beisein ihrer Kinder mit der Mediennutzung verbringen und sie die Medien in einem ungesunden (pathologischen) Ausmass nutzen, wodurch sich dieser Umgang negativ auf das subjektive Wohlbefinden niederschlägt.  

Die Annahme, dass sich Stress auf das Medienverhalten auswirkt, wird in der Studie um \citeA{Mark2014} untersucht. Daraus geht hervor, dass sich Stress auf eine erhöhte Nutzungsdauer vor dem Computer auswirkt und Stress in direktem Zusammenhang mit medialem Multitasking steht. Berufsgruppen, die besonders häufig von Multitasking-Anforderungen betroffen sind, befinden sich in den Bereichen Erziehung und Unterricht und im Gesundheits- und Sozialwesen \cite{Lohmann2012}. Dies lässt darauf hindeuten, dass Eltern, die sich mit der Erziehung der Kinder beschäftigen, besonders vom Multitasking betroffen sind und dementsprechend anfällig für Stress sind. Es wird angenommen, dass während der Betreuung der Kinder viele Aufgaben anfallen, bei denen die Versuchung Multitasking anzuwenden gross ist. Somit könnte eine erhöhte Stressbelastung durch eine erhöhte Ausübung von Multitasking zu einer verstärkten Mediennutzung führen.

Das subjektive Wohlbefinden hängt von den Wünschen und Werten einer Person ab und hat einen subjektiven Charakter \cite{Odaci2014}. Studien konnten belegen, dass eine Person subjektives Wohlbefinden erlebt, wenn ihre Bedürfnisse befriedigt sind und ihre selbstgesteckten Ziele erreicht werden \cite{Diener1999}. Weitere Studien konnten aufdecken, dass das Aufbauen von engen Beziehungen mit dem subjektiven Wohlbefinden einhergeht \cite{Kasser1996} und befriedigende Beziehungen Voraussetzung für das Erreichen eines hohen Grads an Zufriedenheit ist \cite{Celik2013}.

Zusammenfassend lässt sich die Annahme ableiten, dass Eltern, die durch ihre Prädisposition der Bindung und einem hohen aktuellen Stresserleben zu einem verstärkten Medienverhalten tendieren, dadurch eine tendenziell distanzierte Beziehung zu ihren Kindern aufbauen und somit ein reduziertes subjektives Wohlbefinden aufweisen.

\subsection{Hypothesen (Statistische Verfahren)}
\subsubsection{Hypothese 1}
Eltern mit einem sicheren Bindungsstil weisen eine geringere Mediennutzung im Beisein ihrer Kinder auf als Eltern, die einen unsicher-vermeidenden, unsicher-ambivalenten oder desorganisierten Bindungsstil aufweisen.
\subsubsection{Hypothese 2}
Eltern, die ein hohes Ausmass an Stress empfinden, nutzen Medien im Beisein ihrer Kinder häufiger als Eltern, die ein niedriges Ausmass an Stress aufweisen.
\subsubsection{Hypothese 3}
Eltern mit einem erhöhten Medienverhalten im Beisein ihrer Kinder weisen ein geringeres subjektives Wohlbefinden auf als Eltern, die ein geringeres Medienverhalten im Beisein ihrer Kinder aufweisen.
\subsection{Art der Arbeit}
Empirische Querschnitts-Studie mit quantitativem Charakter.

% Theoretischer Hintergrund
% -------------------------
\subsection{Theoretischer Hintergrund / Stand der Forschung}
Seit der Erscheinung des Internets und der Digitalisierung taucht die Frage auf, was für Auswirkungen diese neuen Technologien auf die Benutzer haben. Gemäss heutiger Forschung kann aufgezeigt werden, dass immer mehr jüngere Kinder sich mit dem Internet und den neuen Medien beschäftigen \cite{Rideout2013a, Chaudron2015}, obwohl es ihnen gemäss \citeA{Lobe2011} an technischen, kritischen und sozialen Fähigkeiten mangelt. Dabei stellt sich die Frage, was für Auswirkungen neue Medien auf die Kinder haben \cite{Tomopoulos2010, Pempek2014, Livingstone2015, Masur2015, Troseth2016}. 

Die meisten Umfragen zum Mediennutzungsverhalten von Kindern wurden bei Kindern im Alter zwischen 9 und 16 Jahren durchgeführt \cite{Chaudron2015}. Aus dem amerikanischen Report \citeA{Rideout2013a} geht hervor, dass der Zugang zu mobilen Geräten (iPad) bei Kindern von 8 Jahren und jünger in Amerika gegenüber 2011 von 8\% auf 40\% im Jahre 2013 angestiegen und der Zugang zu einem Smart-Device von 52\% auf 75\% gestiegen ist. Gemäss diesem Report hatten 38\% aller Kinder unter 2 Jahren bereits ein Mobilgerät für die Nutzung von Medien benutzt (gegenüber 10\% in 2011). Studien in der EU kommen in auf ähnliche Ergebnisse \cite{Holloway2013}: Zunahmen von Internetkonsum bei Kindern unter 9 Jahren; Kinder unter 9 Jahren erfreuen sich an diversen online Aktivitäten wie Video schauen, Gamen, Informationssuche, Aufgaben erledigen und mit anderen Kindern sozialisieren; eine Zunahme bei der Verwendung von Geräten mit Touchscreen bei Kindern im Vorschulalter und Kleinkindern ist zu beobachten. Zudem erwähnen die Autoren den digitalen Footprint, der bereits bei sehr kleinen Kindern vorhanden ist. Dieser wird in den meisten Fällen von den eigenen Eltern erzeugt (Fotos teilen, Blogs schreiben, Videos teilen, etc.).

Der Medienkonsum von Kindern und Jugendlichen in der Schweiz wurde durch die ZHAW in der MIKE- und der JAMES-Studie \cite{Suter2015, Waller2016} erhoben. In der MIKE-Studie wurden 2014/2015 Kinder im Primarschulalter (zwischen 6 und 13 Jahren) zu ihrem Medinenutzungsverhalten befragt. Die JAMES-Studie, welche seit 2010 im Zweijahresrythmus durchgeführt wird, erfasste 2016 repräsentative Zahlen zur Mediennutzung von Jugendlichen zwischen 12 und 19 Jahren. Eine qualitative Studie von \citeA{Konitzer2017}, welche im Rahmen des Joint Research Centers (JRC) Projekt ECIT (Empowering Citizens' Rights in emerging ICT) durchgeführt wurde, befragten acht schweizer Familien mit Kindern im Alter zwischen 0 und 8 Jahren bezüglich Gefahren und Nutzen im Umgang mit digitalen Technologien zu Hause. Die Untersucher konnten unter anderem feststellen, dass sich Kinder von verschiedensten digitalen Technologien angezogen fühlen und diese eine wichtige, aber nicht dominante Rolle in ihrem Leben spielen. Weiter konnte festgestellt werden, dass die Eltern eine Vorbildfunktion im angemessenen Umgang mit ICT darstellen. Die Mehrheit ist sich dessen bewusst, finden jedoch nicht immer die Zeit, dies entsprechend umzusetzen. 

Die Studie von \citeA{Livingstone2015a} untersucht den Umgang der Eltern mit digitalen Medien und wie sie diese den Kindern vermitteln. Dabei stellten sie einen Effekt vom soziökonomischen Status, wie Einkommen und Bildung, auf die digitale Mediennutzung im Umgang mit ihren Kindern fest. Länderübergreifende Studien konnten Unterschiede im Verhalten der Eltern im Umgang mit digitalen Medien feststellen \cite{Helsper2013}. Die gemeinsame Nutzung von Medien zwischen Eltern und Kindern wurde unter anderem von \citeA{Livingstone2008, Nikken2014, Plowman2014, Connell2015, Vaala2015, Harrison2015} untersucht. 

Die Auswirkungen von Medienkonsum kann nicht abschliessend beantwortet werden. Es scheint, als ob zum Beispiel die Zeit, die Kinder vor einem Bildschirm sind, abhängig von der Interaktionsfaktoren zwischen Eltern und Kindern ist. Zudem könnte dieses Verhalten in hohem Mass von der Einstellungen der Eltern abhängen \cite{Lauricella2015}. Der direkte Vergleich von einem digitalen Medium (TV) und einem analogen (Buch) zeigte, dass sich die Kommunikation zwischen der Mutter und ihrem lesen lernenden Kind verschlechterte, während ein TV im Hintergrund lief \cite{Nathanson2011}.

Es benötigt weitere Studien, die sich diesem Thema annehmen \cite{Wartella2016}. Die Frage, wie Eltern ihre Kinder bezüglich Kreativität, Lernen und Entwicklung in Bezug zum Medienkonsum prägen, ist unzureichend beantwortet und benötigt weitere Forschung \cite{AmericanAcademyofPediatrics2011,Troseth2016}. 

% Bindungsmuster
\subsubsection{Zusammenhang des Bindungsmusters auf das Medienverhalten} 
Die Familie ist das erste Umfeld, in dem sich ein Kind befindet  und das als Prototyp für zukünftige Beziehungen und Interaktionen fungiert \cite{Floros2013}. Die Familie ist in erster Linie für das Stillen der kindlichen Grundbedürfnisse verantwortlich (nicht nur materiell sondern auch psychologisch). Die psychologischen Bedürfnisse werden durch soziale Bedürfnisse befriedigt \cite{Hazan1994}. Die elterliche Responsiveness dient beim Kind dazu, seine internen Arbeitsmodelle und seinen Bindungsstil gemäss John Bowlby \cite{Bowlby1969} und Mary Ainsworth \cite{Bell1972} zu entwickeln \cite{Bretherton1999}. Ein Kind kann einen sicheren Bindungsstil entwickeln, wenn die Hauptbezugsperson die Signale des Kindes ohne Verzögerung wahrnimmt, richtig interpretiert, angemessen befriedigt und prompt befriedigt \cite{Bell1972}. Studien konnten zeigen, dass sich eine problematische Mutter-Kind-Beziehung auf die spätere Kommunikations- und Beziehungspräferenz niederschlägt \cite{Szwedo2011}. Zudem konnte die Studie von \citeA{Lin2015} aufzeigen, dass der Bindungsstil bei der Nutzung von sozialen Netzwerken ein signifikanter Faktor bezüglich der sozialen Beziehungsorientierung bei Facebook ist und einen Effekt auf das soziale Kapital hat. \citeA{Monacis2017} stellte einen negativen Zusammenhang zwischen sicherem Bindungsstil und Onlinesucht her.  

Gemäss \citeA{Fraley2000} werden im englisch sprachigen Raum vier verbreitete Selbsterfassungsfragebögen für die Erfassung des Bindungsstils (engl. Attachment) eingesetzt: Der \textit{
Experience in Close Relationship scales (ECR)} \cite{Brennan1998}, der \textit{Adult Attachment Scales (AAS)} \cite{Collins1990}, der \textit{Relationship Styles Questionnaire (RSQ)} \cite{Griffin1994} und der von \citeA{Simpson1996} entwickelte \textit{Adult Attachment Questionnaire (AAQ)}. Im deutschsprachigen Raum wird unter anderem die von \citeA{Grau1999} entwickelte \textit{Skala zur Erfassung von Bindungsrepräsentationen} in Paarbeziehungen, die \textit{Beziehungsspezifischen Bindungsskalen für Erwachsene (BBE)} zur Erfassung des partnerschaftlichen Bindungsstils von \citeA{Asendorpf1997} und der ins Deutsche übersetzte \textit{Revised Adult Attachment Scale} von \citeA{Collins1990} eingesetzt. 

% Stress
\subsubsection{Zusammenhang zwischen Stress und Medienverhalten}
Gemäss \citeA{Mark2014} steht Stress in direktem Zusammenhang mit Multitasking (siehe Kapitel Annahmen). Der Begriff Multitasking ist wissenschaftlich noch nicht einheitlich definiert worden. Multitasking steht in der Informatik für die Fähigkeit eines Betriebssystems, verschiedene Aufgaben parallel ausführen zu können \cite{Zimber2016}. Multitasking eignet sich für bestimmte Aufgaben besser als für andere. Je weniger Ressourcen die Parallelaufgabe benötigt und je ähnlicher und automatisierbarer sie sind, desto eher wird Multitasking angewendet \cite[S.~10]{Zimber2016}.

Im Rahmen der Risikofaktoren von Problemverhalten bei Kindern wird häufig vom Faktor Stress bei den Eltern gesprochen, der eng mit dem Verhalten in Erziehungssituationen gezeigt wird und über längere Zeit ungünstige Folgen für das Individuum sowohl dessen Umfeld aufweist \cite{Cina2009}. Psychische, physische sowohl soziale Störungen stehen im Zusammenhang mit Stress \cite{Elfering2002, Burisch1994}. Insbesondere die engen Familienmitglieder sind oft direkt oder indirekt von den Auswirkungen des Stresses betroffen. So zeigen Studien einen Zusammenhang zwischen Stress und schlechtem psychischen Befinden \cite{Burisch1994, Krohne1997}, einer negativen Partneschaftsqualität \cite{Bodenmann2000, Bodenmann1999, Bodenmann2000a} und ungünstigem Erziehungsverhalten \cite{Abidin1992, Belsky1984, WebsterStratton2000}. Tägliche Widrigkeiten scheinen Auswirkungen auf das Erziehungsverhalten der Eltern in Form eines negativen und aversiven Bindungsstils \cite{Dumas1989, Webster-Stratton1988} und einer geringen emotionalen Verfügbarkeit für die Kinder \cite{Campbell1991} zu haben.

Für die Messung des Stressniveaus werden im englischsprachigen Raum drei verbreitete Instrumente eingesetzt \cite{Andreou2011}: Der \textit{Stress Appraisal Measure (SAM)} der \textit{Impact of Event Scale (IES)} und der \textit{Perceived Stress Scale (PSS)}. Der PSS ist für die Messung von Stress weit verbreitet. Hohe PSS Werte gehen mit höheren Cortisolwerten einher, welche als Biomarker für Stress gelten \cite{Malarkey1995, VanEck2005}. Der Test wurde in diverse Sprachen übersetzt und wurde auf dessen Konstruktvalidität geprüft \cite{Cohen1988, Byrne2005}. Im deutschsprachigen Raum wird häufig das Instrument \textit{allgemeines Stressniveau (ASN)} von \citeA{Bodenmann2000} eingesetzt. Dieses Instrument hat sich in diversen Studien bewährt \cite{Cina2009}.

% Methode
% -------
\subsection{Methode}
Bei dieser Masterarbeit handelt es sich um eine Querschnitts-Studie, die mittels Fragebogen erhoben werden soll. 
Es werden deskriptive statistische Verfahren zur Aufdeckung von Datenstrukturen und Abhängigkeitsstrukturen von einer zu rekrutierenden Stichprobe eingesetzt.
% Untersuchungsplan
% -----------------
\subsubsection{Untersuchungsplan / Vorgehen}
TBD: Grafische Übersicht - Was wird wann gemacht.
\begin{enumerate}
    \item Disposition
    \item Literaturrecherche
    \item Zusammenstellen der Umfrage und Tests
    \item Wahl des Umfragetools (Smartphone, Online-Survey)
    \item Rekrutierung der Probanden über einschlägige Organisationen (z.B. Hebammenvereinigung Schweiz, Geburtshäuser, etc.)
    \item Je nach Erhebungsinstrument Einführung der Probanden (z.B. bei Smartphone-Umfrage Konfiguration Startzeitpunkt und Funktions-Kontrolle).
    \item Daten Generierung
    \item Abschluss Umfrage, Bereinigung und Analyse der Rohwerte
    \item Auswertung der Befunde
\end{enumerate}
\subsubsection{Datenerhebung}
Für die Beantwortung der Fragestellung und die Überprüfung der Hypothesen sollen folgende Variablen erhoben werden: 
\begin{seriate}
    \item die Mediennutzung der Eltern im Beisein ihrer bis zu einem Jahr alten Kinder (0-1),
    \item der Bindungsstil der Eltern, 
    \item das Ausmass an Stress der Eltern und
    \item das aktuelle subjektive Wohlbefinden der Eltern
\end{seriate}

Zudem sollen Kovariate wie 
\begin{seriate}
    \item Geschlecht des betreuenden Elternteils und des Kindes
    \item Alter der Eltern und des Kindes
    \item Nationalität
    \item Ausbildung
    \item Einkommen
    \item Beruf und Anstellungsgrad
\end{seriate}
erhoben werden.

Die Erfassung dieser Daten erfolgt mittels Fragebogen, der aus einzelnen bestehenden Fragebögen oder Teilen davon zusammengesetzt wird. Folgende Fragebögen werden für diese Untersuchung verwendet:
\begin{enumerate}
    \item Der Fragebogen zur Erfassung der Mediennutzung wird basierend auf der JAMES-Studie \cite{Waller2016} und Elementen aus dem ECIT-Projekt \cite{Konitzer2017} erstellt. Dabei ist zu beachten, dass eine Unterteilung in online und offline Medien vorgenommen wird. Zudem sollen die vorhandenen digitalen Geräte im Haushalt erfasst werden.
    \item Fragebogen zur Erfassung des persönlichen Bindungsstils der Eltern: 
In dieser Arbeit wird die Skala zur Erfassung von Bindungsrepräsentationen nach \citeA{Grau1999} verwendet. Diese dient zur Klassifizierung der vier Bindungsstile nach \citeA{Bartholomew1990} anhand der beiden Dimensionen Angst und Vermeidung. Der Fragebogen besteht aus zwei intern konsistenten, voneinander unabhängigen Skalen, denen jeweils zehn Items der beiden Dimensionen Angst bzw. Vermeidung zugeordnet sind \cite{Baadte2006}. Zum Antworten dient eine sieben-stufige Skala mit den beiden Polen ‚stimmt nicht‘ (1) bzw. ‚stimmt‘ (7). Alternativ könnte auch der ins Deutsche übersetzte Adult Attachment Scale (AAS) von \citeA{Collins1990} verwendet werden. Dieser, in der revidierten Fassung (AAS-R),  ist über die Testzentrale des Verlags Hogrefe zu beziehen und verfügt über 18 Items zur Erfassung der grundlegenden Dimensionen von Bindung.
    \item Fragebogen zur Erfassung von Stress der Bezugsperson: Für das aktuelle Stressausmass bei den Eltern wird der Fragebogen allgemeines Stressniveau (ASN) von \citeA{Bodenmann2000} verwendet. Auf einer fünfstufigen Skala wird der Aktuelle Belastungsgrad in verschiedenen Bereichen (z.B.: Beruf, Kinder, Partnerschaft, Finanzen, etc.) erhoben. Anschliessend wird der Gesamtwert über die 21 Items erhoben, welches das subjektive Stressniveau im Alltag bildet.
    
    \item Das subjektive Wohlbefinden soll mittels standardisierten Fragebogen, wie zum Beispiel dem \textit{PWI - Personal Wellbeing Index}  \cite{TheInternationalWellBeingGroup2006}, erhoben werden. 
\end{enumerate}

% Stichprobenumfangsplanung g
\subsubsection{Stichprobenumfangsplanung}
Gemäss Fragestellung wird eine Stichprobengrösse von insgesamt 280 Personen angestrebt. Diese Stichprobengrösse wurde mittels G*Power \cite{Faul2009} für eine einfaktorielle Varianzanalyse gemäss \citeA{Rasch2014} ermittelt (F-Test - ANOVA: Fixed effects, omnibus, one-way).
Für die Stichprobenumfangsplanung wird ein mittlerer Effekt \cite{Cohen1988a} als inhaltlich relevant festgelegt: $f = 0.25$ resp. $\Omega^2 = 0.06$. Das Signifikanzniveau beträgt $\alpha=0.05$. Die Teststärke soll mindesten $1-\beta=0.95$ betragen. Es werden vier Gruppen für den Faktor Bindung benötigt.

Für die Beantwortung des Zusammenhangs zwischen Medienverhalten und subjektiven Wohlbefinden wird eine lineare Regressionsanalyse durchgeführt. Das Kriterium bildet das subjektive Wohlbefinden und der Prädiktor das Medienverhalten. Die Stichprobengrösse für die Beantwortung dieser Fragestellung wurde wieder mittels G*Power für eine Korrelation berechnet. Gemäss der Einteilung von \citeA{Cohen1988a} wird ein mittlerer Effekt von $f=.3$ erwartet, was einem Korrealtionskoeffizient $r$ nach Bravais-Pearson von $|r|=.3$ entspricht. Das Signifikanzniveau beträgt $\alpha=0.05$. Die Teststärke soll mindesten $1-\beta=0.95$ betragen. Daraus ergibt sich eine Stichprobengrösse von 138 Probanden.    

% Rekrutierung
\subsubsection{Rekrutierung}
Die Rekrutierung soll über verschiedene einschlägige Kreise im deutschsprachigen Raum erfolgen: 
\begin{seriate}
    \item Über den schweizerischen Hebammenverband (SHV),
    \item den Schweizerischen Fachverband Mütter- und Väterberatung (SF MVB),
    \item die Interessengemeinschaft der Geburtshäuser Schweiz (IGGH-CH),
    \item die Wochenbettabteilung der schweizerischen Spitäler,
    \item Familienzentren in der deutschsprachigen Schweiz und
    \item wenn möglich über Kinderarztpraxen.
\end{seriate}
\paragraph{Einschlusskriterien}
Alle Eltern mit Wohnort in der Schweiz, die mindestens ein Kind zwischen 0 und einem Jahr haben und der deutschen Sprache mächtig sind.
\paragraph{Ausschlusskriterien}
Kinder, die älter als ein Jahr sind, werden nicht berücksichtigt und Eltern, die der deutschen Sprache nicht mächtig sind.
% Statistische Verfahren
\subsubsection{Statistische Verfahren}
Die Auswertung der Daten erfolgt mittels SPSS. 
Für die Beantwortung der Fragestellung, welchen Effekt der Bindungsstil und das aktuelle Stressempfinden der Eltern auf deren Medienverhalten haben, wird eine Einfaktorielle-Varianzanalyse ohne Messwiederholung durchgeführt. Der Faktor Bindungsstil liegt in vier Ausprägungen vor (sicher, unsicher-vermeidend, unsicher-ambivalent und desorganisiert). Der Faktor Stress fliesst in Form einer Kovariate in die Berechnung ein. Es liegt somit eine Varianzanalyse mit 4 Gruppen vor. 

Der Medienkonsum nimmt die Funktion der abhängigen Variable ($AV$) ein und der Bindungsstil die unabhängigen Variable ($UV1$). Es wird ein mittlerer Effekt ($d=0.25$) erwartet. Das Signifikanzniveau wird auf ($\alpha=0.05$) angesetzt und die Teststärke mit ($1-\beta=0.95$) angewendet.

Die Hypothese geht davon aus, dass Eltern mit einem sicheren Bindungsstil und einem geringen Ausmass an Stress ein geringes Medienverhalten aufweisen. Daraus ergibt sich eine gerichtete Hypothese, die gegen die Nullhypothese getestet werden soll.

Der Zusammenhang zwischen dem Medienverhalten und dem subjektiven Wohlbefinden wird mittels Regressionsanalyse ermittelt. Dabei soll ein mittlerer Zusammenhang mittels Produkt-Moment-Korrelationskoeffizienten (Pearson-Korrelations-Koeffizienten $|r|=0.3$) zwischen den einzelnen Variablen nachgewiesen werden.

\subsection{Abgrenzung}
In dieser Arbeit kann die Auswirkung des Medienverhaltens der Eltern auf die Kinder nicht untersucht werden. Durch die begrenzte Mitteilungsmöglichkeiten der Kinder kann die direkte Auswirkung des Medienverhaltens in diesem Rahmen nicht ermittelt werden. Mögliche längerfristige Auswirkungen auf den Bindungsstil der Kinder müsste mittels Langzeitstudie erhoben werden. 
