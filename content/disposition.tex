\section{Beschreibung der Arbeit / Disposition}
\hspace{-0.9cm} % um die Tabelle linksbündig zu haben (ging mit flushleft nicht)
\begin{tabular}{p{4,5cm} p{10cm}} 
Name des Schreibenden: & Till J. ERNST \\ 
Name des Referenten: & Gregor WALLER \\
Vertiefungsrichtung: & Klinische Psychologie \\
& \\
Titel der Arbeit (prov.): & Medienverhalten der Eltern und ihrer Kleinkinder und die Auswirkung auf das Subjektive Wohlbefinden.\\
\end{tabular} \\
% Fragestellung und Annahmen
\subsection{Fragestellung und Annahmen}
\subsubsection{Fragestellung}
\subsubsection{Annahmen}
TBD: Einfluss von Bindungsmuster und Stress auf das Medienverhalten der Eltern und ihrer Kinder und die daraus folgende Auswirkung auf das Subjektive Wohlbefinden der Eltern.

\subsection{Hypothesen (Statistische Verfahren)}
TBD
\subsection{Art der Arbeit}
Empirische Querschnitts-Studie mit quantitativem Charakter.
% Theoretischer Hintergrund
% -------------------------
\subsection{Theoretischer Hintergrund / Stand der Forschung}
\subsubsection{Einführung und allgemeiner Medienumgang Eltern / Kind}
Seit der Erscheinung des Internets und der Digitalisierung taucht die Frage auf, was für Auswirkungen diese neuen Technologien auf die Benutzer haben. Gemäss heutiger Forschung kann aufgezeigt werden, dass immer mehr jüngere Kinder sich mit dem Internet und den neuen Medien beschäftigen \cite{Rideout2013, Chaudron2015}, obwohl es ihnen gemäss \citeA{Lobe2011} an technischen, kritischen und sozialen Fähigkeiten mangelt. Dabei stellt sich die Frage, was für Auswirkungen neue Medien auf die Kinder haben \cite{Tomopoulos2010, Pempek2014, Livingstone2015, Masur2015, Troseth2016}. 

Die meisten Umfragen zum Mediennutzungsverhalten von Kindern wurden im Alter zwischen 9 und 16 Jahren durchgeführt \cite{Chaudron2015}. Aus dem amerikanischen Report \citeA{Rideout2013} geht hervor, dass der Zugang zu mobilen Geräten bei Kindern von 8 Jahren und jünger in Amerika gegenüber 2011 von 8\% beim iPad auf 40\% im Jahre 2013 angestiegen ist. Der Zugang zu einem Smart-Device von 52\% auf 75\% gestiegen ist. Gemäss diesem Report hatten 38\% aller Kinder unter 2 Jahren bereits ein Mobilgerät für die Nutzung von Medien benutzt (gegenüber 10\% in 2011). Studien in der EU kommen in auf ähnliche Ergebnisse \cite{Holloway2013}: Zunahmen von Internetkonsum bei Kindern unter 9 Jahren; Kinder unter 9 Jahren erfreuen sich bei diversen online Aktivitäten wie Video schauen, Gamen, Informationssuche, Aufgaben erledigen und mit anderen Kindern sozialisieren; eine Zunahme bei der Verwendung von Geräten mit Touchscreen bei Kindern im Vorschulalter und Kleinkindern ist zu beobachten. Zudem erwähnen die Autoren den digitalen Footprint, der bereits bei sehr kleinen Kindern vorhanden ist. Dieser wird in den meisten Fällen von den eigenen Eltern erzeugt (Fotos teilen, Blogs schreiben, Videos teilen, etx.).
Der Medienkonsum von Kindern und Jugendlichen in der Schweiz wurde durch die ZHAW in der MIKE- und der JAMES-Studie \cite{Suter2015, Waller2016} erhoben. In der MIKE-Studie wurden 2014/2015 Kinder im Primarschulalter (zwischen 6 und 13 Jahren) zu ihrem Medinenutzungsverhalten befragt. Die JAMES-Studie, welche seit 2010 im Zweijahresrythmus durchgeführt wird, erfasste 2016 repräsentative Zahlen zur Mediennutzung von Jugendlichen zwischen 12 und 19 Jahren. In der Schweiz fehlen Studien im Umgang mit Medien im Bereich Eltern und deren Kleinkinder. Die Studie von \citeA{Livingstone2015a} untersucht den Umgang der Eltern mit digitalen Medien und wie sie diese den Kindern vermitteln wobei der soziökonomische Status, wie Einkommen und Bildung, die digitale Mediennutzung im Umgang mit ihren Kindern beeinflusst. Länderübergreifende Studien konnten Unterschiede im Verhalten der Eltern im Umgang mit digitalen Medien feststellen \cite{Helsper2013}. Die gemeinsame Nutzung von Medien zwischen Eltern und Kindern wurde unter anderem von \citeA{Livingstone2008, Nikken2014, Plowman2014, Connell2015, Vaala2015, Harrison2015} untersucht. 

Die Auswirkungen von Medienkonsum kann nicht abschliessend beantwortet werden. Es scheint, als ob zum Beispiel die Zeit, die Kinder vor einem Bildschirm sind, abhängig von der Interaktionsfaktoren zwischen Eltern und Kindern ist, zusätzlich könnte dies in hohem Mass von der Einstellungen der Eltern abhängen \cite{Lauricella2015}. Der direkte Vergleich von einem digitalen Medium (TV) und einem analogen (Buch) zeigte, dass sich die Kommunikation zwischen der Mutter und ihrem Kind beim Lesen lernen währendem ein TV im Hintergrund läuft verschlechterte \cite{Nathanson2011}.

Es benötigt weitere Studien, die sich diesem Thema annehmen \cite{Wartella2016}. Die Frage, wie Eltern ihr Kinder bezüglich Kreativität, Lernen und Entwicklung im Bezug zum Medienkonsum ist unzureichend beantwortet und benötigt weitere Forschung \cite{AmericanAcademyofPediatrics2011,Troseth2016}. 
\subsubsection{Einfluss der Bindungsmuster auf das Medienverhalten}
TBD
\subsubsection{Einfluss von Stress auf das Medienverhalten}
TBD
\subsubsection{Auswirkung Medienverhalten auf das SWB}
TBD

% Methode
% -------
\subsection{Methode}
Bei dieser Masterarbeit handelt es sich um eine Querschnitts-Studie, in der mittels Fragebogen die Mediennutzung, der Bindungsstil und der aktuellen Stresslevel des betreuenden Elternteils empirisch erfasst werden soll. 
Es werden deskriptive statistische Verfahren zur Aufdeckung von Datenstrukturen und Abhängigkeitsstrukturen von einer zu rekrutierenden Stichprobe eingesetzt.
% Untersuchungsplan
% -----------------
\subsubsection{Untersuchungsplan / Vorgehen.}
TBD: Was wird wann gemacht.
\subsubsection{Datenerhebung}
TBD: siehe \cite{AmericanAcademyofPediatrics2013} und \cite{Plowman2014}, \cite{Nikken2014}
\subsubsection{Stichprobe und Rekrutierung}
\subsubsection{Statistische Verfahren}
\subsubsection{Weitere Bedingungen}
\subsection{Abgrenzung}
