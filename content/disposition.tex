\section{Beschreibung der Arbeit / Disposition}
\hspace{-0.9cm} % um die Tabelle linksbündig zu haben (ging mit flushleft nicht)
\begin{tabular}{p{4,5cm} p{10cm}} 
Name des Schreibenden: & Till J. ERNST \\ 
Name des Referenten: & Gregor WALLER \\
Vertiefungsrichtung: & Klinische Psychologie \\
& \\
Titel der Arbeit (prov.): & Medienverhalten der Eltern und ihrer Kleinkinder und die Auswirkung auf das Subjektive Wohlbefinden.\\
\end{tabular} \\
% Fragestellung und Annahmen
\subsection{Fragestellung und Annahmen}
\subsubsection{Fragestellung}
\subsubsection{Annahmen}
TBD: Einfluss von Bindungsmuster und Stress auf das Medienverhalten der Eltern und die daraus folgende Auswirkung auf das Subjektive Wohlbefinden.

\subsection{Hypothesen (Statistische Verfahren)}
TBD
\subsection{Art der Arbeit}
Empirische Querschnitts-Studie mit quantitativem Charakter.
% Theoretischer Hintergrund
% -------------------------
\subsection{Theoretischer Hintergrund / Stand der Forschung}
Seit der Erscheinung des Internets und der Digitalisierung taucht die Frage auf, was für Auswirkungen diese neuen Technologien auf die Benutzer hat. Gemäss heutiger Forschung kann aufgezeigt werden, dass immer mehr jüngere Kinder sich mit dem Internet und den neuen Medien beschäftigen \cite{Chaudron2015}, obwohl es ihnen gemäss \citeA{Lobe2011} an technischen, kritischen und sozialen Fähigkeiten mangelt.
Die meisten Umfragen zum Mediennutzungsverhalten von Kindern wurden im Alter zwischen 9 und 16 Jahren durchgeführt \cite{Chaudron2015}. Der Medienkonsum von Kindern und Jugendlichen in der Schweiz wurde durch die ZHAW in der MIKE- und der JAMES-Studie \cite{Suter2015, Waller2016} erhoben. In der MIKE-Studie wurden 2014/2015 Kinder im Primarschulalter (zwischen 6 und 13 Jahren) zu ihrem Medinenutzungsverhalten befragt. Die JAMES-Studie, welche seit 2010 im Zweijahresrythmus durchgeführt wird, erfasste 2016 repräsentative Zahlen zur Mediennutzung von Jugendlichen zwischen 12 und 19 Jahren. 
Dabei stellt sich die Frage, was für Auswirkungen neue Medien auf die Kinder haben \cite{Livingstone2015}.  


% Methode
% -------
\subsection{Methode}
Bei dieser Masterarbeit handelt es sich um eine Querschnitts-Studie, in der mittels Fragebogen die Mediennutzung, der Bindungsstil und der aktuellen Stresslevel des betreuenden Elternteils empirisch erfasst werden soll. 
Es werden deskriptive statistische Verfahren zur Aufdeckung von Datenstrukturen und Abhängigkeitsstrukturen von einer zu rekrutierenden Stichprobe eingesetzt.
% Untersuchungsplan
% -----------------
\subsubsection{Untersuchungsplan / Vorgehen.}
TBD: Was wird wann gemacht.
\subsubsection{Datenerhebung}
TBD: 
\subsubsection{Stichprobe und Rekrutierung}
\subsubsection{Statistische Verfahren}
\subsubsection{Weitere Bedingungen}
\subsection{Abgrenzung}
